\documentclass[11p]{article}
% Packages
\usepackage{amsmath}
\usepackage{graphicx}
\usepackage{fancyheadings}
\usepackage[swedish]{babel}
\usepackage[
    backend=biber,
    style=authoryear-ibid,
    sorting=ynt
]{biblatex}
\usepackage[utf8]{inputenc}
\usepackage[T1]{fontenc}
%Källor
\addbibresource{references.bib}
\graphicspath{ {./images/} }

% Lite variabler
\def\email{endo.axelsson@ga.ntig.se}
\def\foottitle{PMmall}
\def\name{Endo Axelsson}

\title{PMmall \\ \small Gymnasiearbete}
\author{\name}
\date{\today}

\begin{document}

% fixar sidfot
\lfoot{\footnotesize{\name \\ \email}}
\rfoot{\footnotesize{\today}}
\lhead{\sc\footnotesize\foottitle}
\rhead{\nouppercase{\sc\footnotesize\leftmark}}
\pagestyle{fancy}
\renewcommand{\headrulewidth}{0.2pt}
\renewcommand{\footrulewidth}{0.2pt}

% i Sverige har vi normalt inget indrag vid nytt stycke
\setlength{\parindent}{0pt}
% men däremot lite mellanrum
\setlength{\parskip}{10pt}

\maketitle


\section{Introduktion}
Denna undersökning handlar om hur den kognitiva förmågan och arbetsminnet presterar med samband till musiken.
Musik som på så sätt är bekant eller lätt att återkalla kan lugna ner stressnivån för att med fördel plocka upp information enklare.
Det som undersökts är om musiken med hjälp av ett memory spel kan hålla informationen som plockas upp aktuellt i minnet.
Är musiken en underlättande metod för den kognitiva förmågan.
Jämväl om den hjälper inlärningen och ökar motivationen.
Kognitiva förmågan som begrepp handlar om hur hjärnan under en kort period tar upp information medvetet.
För att sedan lagra det, bearbetar och plocka fram informationen för senare scenarion.


\section{Bakrund}
Hörlurar med ljud i öronen är ett alltmer populärt val gällande studieteknik.
Det blockerar ut allt oljud runtomkring och ersätter det med harmoniska ljud som lugnar ner hjärnans stressnivå.
För andra höjer den dopaminet, vilket också översätts till att humöret blir bättre.
Hjärnan blir bättre inställd och är mer motiverad för att ta emot fakta.
Denna studie undersöker hur information lagras i minnet i sammanhang med musik och om man kan utnyttja det som en hjälpmedel för att återkalla minnerna.
Som \textcite{Effectsmusic} nämner så är det som när väldigt unga barn blir inlärda med hjälpmedel som musiken.
Låtar som Alfabetssången och tvätta händerna är bland de vanligaste.
Det lär in barn alfabetet och att man ska tvätta händerna med en glad och en lätt igenkänlig melodi.
Därtill finns det musik videor där dem visuellt associerar det som visas på skärmen med ett ord eller meningar.
Grejer som bus, färger, djur m.m.
Lyssnar man om låtarna flertal gånger så får hjärnan små signaler från tidigare tillfällen och på så sätt återkalla det som är kopplat till det specifika situationen.
Mera upprepande innebär att informationen blir alltmer aktuella och bekantare för minnet att hålla kvar.
Det har många faktorer som kan anting försämra eller förbättra detta, som vilken genre musiken är eller personens koncentrationsnivå ligger på.


\subsection{Arbetsminnet}
noteringar:
\begin{itemize} Arbetsminnet är en process i hjärnan där den medvetet plockar fram och manipulerar relevanta information. \end{itemize}

\begin{itemize} Hjärnan har flertal regioner inom sig där olika minnen lagras. \end{itemize}
\begin{itemize} Det finns olika aspekter i hjärnan som stimuleras beroende på scenario. \end{itemize}
\begin{itemize}  Men genom musiken ska minnet få en bättre miljö för att lagra det visuella och informativa. \end{itemize}
\begin{itemize}  Denna information är i mindre mängder och är dessutom lika långvarig. \end{itemize}
\begin{itemize} Informationen är lagrad i arbetsminnet under en kort period tills repetition uppstår.  \end{itemize}




\end{document}
